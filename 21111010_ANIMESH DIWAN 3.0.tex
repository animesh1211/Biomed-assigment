\documentclass[12pt]{article}

\title{National Institute of Technology, Raipur}



\author{ASSIGMENT-1}


\date{[jan 28, 2022]}
\usepackage{indentfirst}
\usepackage{graphicx}


\begin{document}
\maketitle
\centering
\title{WRIGHT-UP ON ANY 5 MEDICAL DEVICES}


\includegraphics[scale=0.8]{O:/desktop 2.0/latex.comm/imagephoto.latex/nitrr logo.png}
\centering

submitted by,
Name:-Animesh Diwan\\
Roll no. :- 21111010\\
Branch:- BIOMADICAL\\
\newpage


\tableofcontents

 

\section{MAGNETIC RESONANCE IMAGING,DEVICE}
\includegraphics[scale=1]{O:/desktop 2.0/latex.comm/imagephoto.latex/MRI.jpg}

 

\subsection{What is a MRI device}

Magnetic resonance imaging (MRI) is a medical imaging technique used in radiology to form pictures of the anatomy and the physiological processes of the body.MRI is a medical application of nuclear magnetic resonance (NMR) which can also be used for imaging in other NMR applications, such as NMR spectroscopy. MRI was originally called NMRI (nuclear magnetic resonance imaging), but "nuclear" was dropped to avoid negative associations.


\subsection{Functioning or Mechanism}

 MRI scanners use strong magnetic fields, magnetic field gradients, and radio waves to generate images of the organs in the body .hydrogen nuclei, which consist solely of a proton, that are in tissues create a signal that is processed to form an image of the body in terms of the density of those nuclei in a specific region. 
 
 Given that the protons are affected by fields from other atoms to which they are bonded, it is possible to separate responses from hydrogen in specific compounds. To perform a study, the person is positioned within an MRI scanner that forms a strong magnetic field around the area to be imaged.


\subsection{Major uses}

MRI is widely used in hospitals and clinics for medical diagnosis, staging and follow-up of disease. Compared to CT, MRI provides better contrast in images of soft-tissues, e.g. in the brain or abdomen. However, it may be perceived as less comfortable by patients, due to the usually longer and louder measurements with the subject in a long, confining tube.

 Additionally, implants and other non-removable metal in the body can pose a risk and may exclude some patients from undergoing an MRI examination safely.MRI has a wide range of applications in medical diagnosis and more than 25,000 scanners are estimated to be in use worldwide.[15] MRI affects diagnosis and treatment in many specialties although the effect on improved health outcomes is disputed in certain cases.

\includegraphics[scale=1]{O:/desktop 2.0/latex.comm/imagephoto.latex/MRI2.jpg}

 
\subsection{Various medical sector benefits}

MRI is the investigative tool of choice for neurological cancers over CT, as it offers better visualization of the posterior cranial fossa, containing the brainstem and the cerebellum.\newline
 \indent Cardiac MRI is complementary to other imaging techniques, such as echocardiography, cardiac CT, and nuclear medicine. It can be used to assess the structure and the function of the heart.\newline 
 \indent Hepatobiliary MR is used to detect and characterize lesions of the liver, pancreas, and bile ducts. Focal or diffuse disorders of the liver may be evaluated using diffusion-weighted, opposed-phase imaging and dynamic contrast enhancement sequences.\newline 
 \indent Magnetic resonance angiography (MRA) generates pictures of the arteries to evaluate them for stenosis (abnormal narrowing) or aneurysms (vessel wall dilatations, at risk of rupture). MRA is often used to evaluate the arteries of the neck and brain, the thoracic and abdominal aorta, the renal arteries, and the legs (called a "run-off"). 










\section{Dental Radiography}


\subsection{Introduction}Dental radiographs are commonly called X-rays.Dental X-rays (radiographs) are images of your teeth that your dentist uses to evaluate your oral health. These X-rays are used with low levels of radiation to capture images of the interior of your teeth and gums. This can help your dentist to identify problems, like cavities, tooth decay, and impacted teeth.


\subsection{Mechanism and Functioning}A radiographic image is formed by a controlled burst of X-ray radiation which penetrates oral structures at different levels, depending on varying anatomical densities, before striking the film or sensor. Teeth appear lighter because less radiation penetrates them to reach the film.The dosage of X-ray radiation received by a dental patient is typically small (around 0.150 mSv for a full mouth series
\newpage

 

\subsection{Medical uses} Dentists use radiographs for many reasons: to find hidden dental structures, malignant or benign masses, bone loss, and cavities.\newline 
Digital X-rays, which replace the film with an electronic sensor, address some of  issues which have on traditional x-rays, and are becoming widely used in dentistry as the technology evolves. They may require less radiation and are processed much more quickly than conventional radiographic films, often instantly viewable on a computer.


\begin{figure}{b}

\centering
\includegraphics[scale=0.7]{O:/desktop 2.0/latex.comm/imagephoto.latex/dental xray.jpg}
\caption{DENTAL X-RAY MACHINE}
\end{figure}



\subsection{Risk of dental x-ray} 

 Repeated exposure to dental X-rays may increase the risk of thyroid cancer and tumours in tissue covering the brain and spinal cord, according to new research.About 3,500 new cases of thyroid cancer and 1,850 'meningiomas' (tumours which are mostly benign and grow slowly) are diagnosed each year in the UK and researchers have discovered an increase in both diseases in many countries in the past 30 years.



\section{ Carbon Dioxide Sensor or CO2 Sensor}
\subsection{Carbon Dioxide Sensor}

A carbon dioxide sensor or CO2 sensor is an instrument for the measurement of carbon dioxide gas.

\includegraphics[scale=1]{O:/desktop 2.0/latex.comm/imagephoto.latex/co2.jpg} 
 


\subsubsection{Affects of Carbon Dioxide}

 Exposure to too much carbon dioxide (above 1000 ppm) can have adverse effects on both your brain and sleep.

\subsection{Functioning}

The most common principles for CO2 sensors are infrared gas sensors (NDIR) and chemical gas sensors.

\includegraphics[scale=0.5]{O:/desktop 2.0/latex.comm/imagephoto.latex/co2 1.jpg}


\subsubsection{Nondispersive Infrared (NDIR) CO2 Sensorsr}

NDIR sensors are spectroscopic sensors to detect CO2 in a gaseous environment by its characteristic absorption. The key components are an infrared source, a light tube, an interference (wavelength) filter, and an infrared detector. The gas is pumped or diffuses into the light tube, and the electronics measure the absorption of the characteristic wavelength of light.\newline
 
\subsubsection{Chemical CO2 Sensors}


Chemical CO2 gas sensors with sensitive layers based on polymer- or heteropolysiloxane have the principal advantage of very low energy consumption, and that they can be reduced in size to fit into microelectronic-based systems



\subsubsection{Estimated CO2 Sensor}


Sensors of that  can be made using cheap  MEMS metal oxide semiconductor (MOS) technology. The reading they generate is called estimated CO2 (eCO2) or CO2 equivalent (CO2eq).For indoor environments such as offices or gyms where the principal source of CO2 is human respiration, rescaling some easier-to-measure quantities such as volatile organic compound (VOC) and hydrogen gas (H2) concentrations provides a good-enough estimator of the real CO2 concentration for ventilation and occupancy purposes.




\subsection{Commonly Uses}

Measuring carbon dioxide is important in monitoring indoor air quality, the function of the lungs in the form of a capnograph device, and many industrial processes.



\subsection{Purpose for use}Why we Should Use a Carbon Dioxide (CO2) Monitor\newline
Because carbon dioxide has no color or smell, it often goes unnoticed in our home. Even though you don’t see it doesn’t mean that it can’t hurt you.Exposure to too much carbon dioxide (above 1000 ppm) can have adverse effects on both your brain and sleep








\section{"Thermal Radiation Thermometers",Laser Thermometers or Infrared Thermometer }
\subsection{Introduction}

An infrared thermometer is a thermometer which infers temperature from a portion of the thermal radiation sometimes called black-body radiation emitted by the object being measured

\includegraphics[scale=1]{O:/desktop 2.0/latex.comm/imagephoto.latex/infrared thermo.jpg} 


\subsection{Mechanism or Functioning}

By knowing the amount of infrared energy emitted by the object and its emissivity, the object's temperature can often be determined within a certain range of its actual temperature.The design essentially consists of a lens to focus the infrared thermal radiation on to a detector, which converts the radiant power to an electrical signal that can be displayed in units of temperature after being compensated for ambient temperature. This permits temperature measurement from a distance without contact with the object to be measured.






\subsection{Benefits in various Condition}

 laser thermometers as a laser is used to help aim the thermometer, or non-contact thermometers or temperature guns, to describe the device's ability to measure temperature from a distance.Sometimes, especially near ambient temperatures, readings may be subject to error due to the reflection of radiation from a hotter body—even the person holding the instrument[citation needed] — rather than radiated by the object being measured, and to an incorrectly assumed emissivity.
 
 \includegraphics[scale=1]{O:/desktop 2.0/latex.comm/imagephoto.latex/termometer.jpg} 
 
\subsubsection{accuracy}


Infrared thermometers are characterized by specifications including accuracy and angular coverage. Simpler instruments may have a measurement error of about ±2 °C or ±4 °F.



\section{ENDOSCOPY} 


a procedure in which an instrument is introduced into the body to give a view of its internal parts.


\includegraphics[scale=1]{O:/desktop 2.0/latex.comm/imagephoto.latex/endoscopy.jpg} 




\subsection{Endoscopty}

An endoscopy  is a procedure used in medicine to look inside the body. The endoscopy procedure uses an endoscope to examine the interior of a hollow organ or cavity of the body. Unlike many other medical imaging techniques, endoscopes are inserted directly into the organ


\subsection{How it can used?}

The doctor uses a tool called an endoscope to do an upper endoscopy. An endoscope is a thin, flexible tube with a light and a tiny camera on the end. The doctor inserts it into the mouth, down the throat, and into the esophagus. The doctor views the images on a screen to look for tumors or other health problems.An endoscopy is not usually painful, but it can be uncomfortable. Most people only have mild discomfort, similar to indigestion or a sore throat. The procedure is usually done while you're awake. You may be given a local anaesthetic to numb a specific area of your body.



\subsection{General Medical Uses }

Endoscopy may be used to investigate symptoms in the digestive system including nausea, vomiting, abdominal pain, difficulty swallowing, and gastrointestinal bleeding.It is also used in diagnosis, most commonly by performing a biopsy to check for conditions such as anemia, bleeding, inflammation, and cancers of the digestive system.



\subsection{Some Complication in Endoscopy}

Possible complications of endoscopy include: Perforation of an organ.\newline
Excessive bleeding (haemorrhage)\newline
Infection.\newline
Allergic reaction to the anaesthesia.\newline
Inflammation of the pancreas (pancreatitis) after an ERCP.
\subsubsection{Disadvantages of Endoscopy}


Endoscopy cannot detect functional disease of the GI tract or estimate luminal diameter as well as other techniques. 2, 6 Motility disorders of the intestine, such as esophageal dysmotility or irritable bowel syndrome, cannot be detected via endoscopy.



\end{document}ss