\documentclass[12pt]{article}

\usepackage{pgf}
\usepackage{pgfpages}

\pgfpagesdeclarelayout{boxed}
{
  \edef\pgfpageoptionborder{0pt}
}
{
  \pgfpagesphysicalpageoptions
  {%
    logical pages=1,%
  }
  \pgfpageslogicalpageoptions{1}
  {
    border code=\pgfsetlinewidth{3.5pt}\pgfstroke,%
    border shrink=\pgfpageoptionborder,%
    resized width=.95\pgfphysicalwidth,%
    resized height=.95\pgfphysicalheight,%
    center=\pgfpoint{.5\pgfphysicalwidth}{.5\pgfphysicalheight}%
  }%
}

\pgfpagesuselayout{boxed}
\begin{document}



\title{National Institute of Technology, Raipur}



\author{ASSIGMENT-5}


\date{[March 04,2022]}





\maketitle
\centering
\title{\textsc{\large    Two page write-up on "5 Solutions to Covid19 Provided by biomedical Engineers"} }


\includegraphics[scale=0.8]{O:/desktop 2.0/latex.comm/imagephoto.latex/nitrr logo.png}
\centering

\begin{flushleft}	
\textsc{\large submitted by:}\\   
\author{Animesh Diwan}\\
[0.2cm]
Roll no : 21111010\\
[0.2cm]
Branch: Biomadical\\
[0.4cm]
\end{flushleft}       
\begin{flushright}
\textsc{\large Under the supervision of :}\\
Mr. Saurabh Gupta \\
Department of Biomadical Engineering\\
\end{flushright}


\newpage


\tableofcontents

\pagebreak 

\begin{flushleft}

\begin{tiny}
\textsc{  " Coronaviruses (CoV) are a broad family of viruses that can cause illnesses ranging from the common cold to more serious illnesses like Middle East Respiratory Syndrome (MERS-CoV) and Severe Acute Respiratory Syndrome (SARS) (SARS-CoV). A novel coronavirus (nCoV) is a new strain of coronavirus that has never been seen in humans before. Coronaviruses are zoonotic, meaning they can spread from animals to humans. SARS-CoV was transmitted from civet cats to humans, and MERS-CoV was transmitted from dromedary camels to humans, according to detailed examinations. Several coronaviruses that have not yet infected people are circulating in animals."}

\end{tiny}










\section{
\textsc{\large Problems faces during  covid19 Pandemic   and their solutions} }
\subsection{patients' oxygen supply}
The delivery of additional oxygen using a nasal cannula or a more intrusive face mask is usually the primary form of treatment for mild respiratory insufficiency. The oxygen is usually delivered in cylinders, which are either tiny for transportation or big for fixed patients and longer-term supplies.\\
[1cm]
Although oxygen concentrators are an appealing option to tanks, they are rarely used in hospital settings for caring for COVID-19 patients. Oxygen concentrators take oxygen from the air and deliver it to the patient on demand. Concentrators are available in a variety of sizes, ranging from a small portable shoulder bag to larger fixed units for patients who require oxygen 24 hours a day. \\
[0.75cm]
\subsection{Ventilators that can save our  life}
Patients who are unable to breathe on their own must be placed on a ventilator. Patients in an advanced stage of respiratory distress are frequently intubated and sedated at the start of treatment since ventilators can replace breath function.\\
[0.9cm]
Patients in an advanced stage of respiratory distress are frequently intubated and sedated at the start of treatment since ventilators can replace breath function. They are complicated devices that give healthcare providers a lot of flexibility in terms of adjusting assisted breathing settings and eventually weaning healing patients off the ventilator. \\
\subsection{Patient monitoring}
The monitoring equipment, which keeps track of some of the patient's vitals, especially when they are ventilated and sedated, but also during their recovery phase to ensure the ventilation regime is optimised for their condition, is an important part of the ICU equipment. Ventilators already have their own set of patient parameters, but patient monitors are usually distinct devices because they are still relevant when the patient can breathe on their own.\\
[0.9cm]
The amount of oxygen in a patient's bloodstream (SpO2), which is evaluated by pulse oximetry, which uses optics within a finger clamp, is one of the most important metrics for COVID-19 patients. Pulse oximetry is often used for the duration of a patient's stay in the intensive care unit. Modern patient monitors provide a plethora of additional patient parameters, all the way down to breathing waveforms, allowing doctors to fine-tune their patient treatment.\\
[.75cm]
\subsection{Personal protective equipment}
The COVID-19 epidemic has highlighted society's vulnerability and the necessity for comprehensive and practical protection measures. Face masks, as personal protection equipment (PPE), remain the greatest practicable line of defence against SARS-CoV-2 and other respiratory virus illnesses for the general public.\\
[0.9cm]
However, greater protection is required for a wide range of interdisciplinary health care personnel, such as surgical or respirator masks, which are not designed to be worn for as long as an NHS shift requires. These disposable objects have an environmental cost, do not fit all face shapes, the mask-face seal can be broken while talking, and put pressure on the sensitive face skin, causing discomfort and tissue injury. They also mask the face, which disadvantages persons with hearing problems who rely on lip reading - as well as a human face being reassuring - from the patients' and carers' viewpoints.\\
[0.75cm]




\subsection{COVID-19 IR Smart Scan Technology}


\includegraphics[scale=0.35]{IR-Smart-Scan-.jpg} 
\centering

IR Smart Scanning Services at Health Screening Stations Thermal imaging technology allows a single person to watch huge groups of persons entering your business efficiently and precisely. The thermal imaging IR Smart Scan technology analyses body temperature in real time as entrants walk past in a systematic manner, allowing personnel with normal temperatures to enter the facility swiftly and cutting down on wait times. Total Safety's IR Smart Scan equipment, as well as our skilled experts, will assist you in preventing COVID-19 from entering your building..



\end{flushleft}







'


\end{document}ss